\documentclass[10pt, letterpaper]{article}
\usepackage{graphicx}
\usepackage{amsmath}
\usepackage[table]{xcolor}
\usepackage{color}
\setlength{\parindent}{0pt}
\setlength{\parskip}{5pt}
\usepackage{textpos}

% use a larger page size; otherwise, it is difficult to have complete
% code listings and output on a single page
\usepackage{fullpage}


% use the listings package for code snippets. define keywords for prm files
% and for gnuplot
\usepackage{listings}
\lstset{frame=tb,
  language=Fortran,
  aboveskip=3mm,
  belowskip=3mm,
  showstringspaces=false,
  columns=flexible,
  basicstyle={\small\ttfamily},
  numbers=none,
  numberstyle=\tiny\color{gray},
  keywordstyle=\color{blue},
  commentstyle=\color{dkgreen},
  stringstyle=\color{mauve},
  breaklines=true,
  breakatwhitespace=true,
  tabsize=3
}

\lstdefinelanguage{prmfile}{morekeywords={set,subsection,end},
                            morecomment=[l]{\#},escapeinside={\%\%}{\%},}
\lstdefinelanguage{gnuplot}{morekeywords={plot,using,title,with,set,replot},
                            morecomment=[l]{\#},}


% use the hyperref package; set the base for relative links to
% the top-level Rayleigh directory so that we can link to
% files in the Rayleigh tree without having to specify the
% location relative to the directory where the pdf actually
% resides
\usepackage[colorlinks,linkcolor=blue,urlcolor=blue,citecolor=blue,destlabel=true,baseurl=../]{hyperref}

\definecolor{dkgreen}{rgb}{0,0.6,0}
\definecolor{gray}{rgb}{0.5,0.5,0.5}
\definecolor{mauve}{rgb}{0.58,0,0.82}

\newcommand{\rayleigh}{\textsc{Rayleigh}}

\begin{document}

\definecolor{dark_grey}{gray}{0.3}
\definecolor{aspect_blue}{rgb}{0.3125,0.6875,0.9375}

%LINE 1%
{
\renewcommand{\familydefault}{\sfdefault}

%\pagenumbering{gobble}

\pagenumbering{arabic}
\renewcommand{\abstractname}{Overview}


\title{Pycnoclinic Implementation \\ Within Rayleigh}
\author{Nick Featherstone}
\date{}
\maketitle

\input definitions.tex

%\tableofcontents

%\input installation.tex

%\input running.tex

%\input benchmarking.tex

\input base_equations.tex

\input ordering.tex

\input thermo.tex

\input first_order.tex

\input filtered.tex

%\input checkpointing.tex

%\input cookbooks.tex

%\input diagnostics.tex

%\input io_redirection.tex

%\input ensemble_mode.tex



\end{document}

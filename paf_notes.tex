\documentclass[10pt, letterpaper]{article}
\usepackage{graphicx}
\usepackage{amsmath}
\usepackage[table]{xcolor}
\usepackage{color}
\setlength{\parindent}{0pt}
\setlength{\parskip}{5pt}
\usepackage{textpos}

% use a larger page size; otherwise, it is difficult to have complete
% code listings and output on a single page
\usepackage{fullpage}


% use the listings package for code snippets. define keywords for prm files
% and for gnuplot
\usepackage{listings}
\lstset{frame=tb,
  language=Fortran,
  aboveskip=3mm,
  belowskip=3mm,
  showstringspaces=false,
  columns=flexible,
  basicstyle={\small\ttfamily},
  numbers=none,
  numberstyle=\tiny\color{gray},
  keywordstyle=\color{blue},
  commentstyle=\color{dkgreen},
  stringstyle=\color{mauve},
  breaklines=true,
  breakatwhitespace=true,
  tabsize=3
}

\lstdefinelanguage{prmfile}{morekeywords={set,subsection,end},
                            morecomment=[l]{\#},escapeinside={\%\%}{\%},}
\lstdefinelanguage{gnuplot}{morekeywords={plot,using,title,with,set,replot},
                            morecomment=[l]{\#},}


% use the hyperref package; set the base for relative links to
% the top-level Rayleigh directory so that we can link to
% files in the Rayleigh tree without having to specify the
% location relative to the directory where the pdf actually
% resides
\usepackage[colorlinks,linkcolor=blue,urlcolor=blue,citecolor=blue,destlabel=true,baseurl=../]{hyperref}

\definecolor{dkgreen}{rgb}{0,0.6,0}
\definecolor{gray}{rgb}{0.5,0.5,0.5}
\definecolor{mauve}{rgb}{0.58,0,0.82}

\newcommand{\rayleigh}{\textsc{Rayleigh}}

\begin{document}

\definecolor{dark_grey}{gray}{0.3}
\definecolor{aspect_blue}{rgb}{0.3125,0.6875,0.9375}

%LINE 1%
{
\renewcommand{\familydefault}{\sfdefault}

%\pagenumbering{gobble}











\pagenumbering{arabic}
\renewcommand{\abstractname}{Overview}


\title{Rayleigh User Guide \\ Version 0.9.1}
\author{Nicholas Featherstone}

\maketitle

\begin{abstract}

Rayleigh solves the MHD equations, in a rotating frame, within spherical shells, using the anelastic or Boussinesq approximations.  Derivatives in Rayleigh are calculated using a spectral transform scheme.  Spherical harmonics are used as basis functions in the horizontal direction.  Chebyshev polynomials are employed in radius.  Time-stepping is accomplished used the semi-implicit Crank-Nicolson method for the linear terms, and the Adams-Bashforth method for the nonlinear terms.  Both methods are second-order in time.
\\
\\
This document serves as a guide to installation and running Rayleigh.   Rayleigh's diagnostics package is discussed in the companion document Rayleigh/doc/Diagnostics\_Plotting.{html,pdf}
\\
\\
Rayleigh is written by Nicholas Featherstone, with National-Science-Foundation support through the Geodynamo Working Group of the Computational Infrastructure for Geodynamics (CIG; PI: Louise Kellogg).
\\
\\
The CIG Geodynamo Working Group Members are:
Jonathon Aurnou, Benjamin Brown, Bruce Buffett, Nicholas Featherstone, Gary Glatzmaier, Eric Heien, Moritz Heimpel, Lorraine Hwang, Louise Kellogg, Hiroaki Matsui, Peter Olson, Krista Soderlund, Sabine Stanley, Rakesh Yadav.
\\
\\
\noindent Rayleigh's implementation of the pseudo-spectral algorithm and its parallel design would not have been possible without earlier work by Gary Glatzmaier and Thomas Clune, described in:
\begin{enumerate}
\item Glatzmaier, G.A., 1984, \textit{J. Comp. Phys.}, 55, 461
\item Clune, T.C., Elliott, J.R., Miesch, M.S. \& Toomre, J., 1999, \textit{Parallel Comp.}, \textbf{25}, 361
\end{enumerate}

\end{abstract}
\clearpage

\tableofcontents

%\input installation.tex

%\input running.tex

%\input benchmarking.tex

\input physics.tex

%\input checkpointing.tex

%\input cookbooks.tex

%\input diagnostics.tex

%\input io_redirection.tex

%\input ensemble_mode.tex



\end{document}

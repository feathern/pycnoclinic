\section{Thermodynamic Relations}\label{sec:thermo}
We treat the plasma as a monatomic ideal gas, so that 
\begin{equation}
P=R\rho T,
\end{equation}
and
\begin{equation}
\frac{S}{\cp} = \mathrm{ln}\left(\frac{P^{1/\gamma}}{\rho} \right)+\frac{5}{2},
\end{equation}
where $S$ is the entropy of the gas, $\cp$ is the specific heat and constant pressure, and $\gamma=\cp/\cv$.  Differentiating with respect to $x$ (say) then yields
\begin{equation}
\frac{1}{P}\frac{\partial P}{\partial x} = \frac{1}{T}\frac{\partial T}{\partial x}+\frac{1}{\rho}\frac{\partial \rho}{\partial x},
\end{equation} 
and
\begin{equation}
\frac{1}{\cp}\frac{\partial S}{\partial x} = \frac{1}{\gamma P }\frac{\partial P}{\partial x} - \frac{1}{\rho}\frac{\partial \rho}{\partial x}.
\end{equation}
For adiabatic changes, we have that
\begin{equation}
\label{eq:adiabatic}
\frac{1}{T}\frac{\partial T}{\partial x} = 
\frac{\gamma-1}{\rho}\frac{\partial \rho}{\partial x} =
\frac{\gamma-1}{\gamma P}\frac{\partial P}{\partial x}.
\end{equation}
Among other things, this implies that temporal variations in $T$, $\rho$, and $P$ will occur in-phase for the first-order adiabatic oscillations associated with our sound wave.

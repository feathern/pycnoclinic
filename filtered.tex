\section{Filtered Equations}
\subsection{Temporal Filtering Procedure}
Our higher-order dynamics will be obtained by temporal filtering of the flow, much as in Koulakis et al.   Let $\tau$ be the acoustic period and $\overline{\tau}$ be timescale characteristic of second-order variations.   We filter by averaging in time over an intermediate timescale $n\tau$, with $n\sim O(1)$, so that $n\tau$ is still substantially faster than the slow timescale.  We use brackets or overbars to denote the average, so that for any quantity $f$, we have
\begin{equation}
\overline{f} = \langle f \rangle \equiv \frac{1}{n\tau}\int_{t-n\tau/2}^{t+n\tau/2} f(x)\, \partial x
\end{equation}
For oscilliatory quantities, we then have
\begin{equation}
\overline{f}_1 \approx 0,
\end{equation}
while for slowly varying quantities, we have that
\begin{equation}
\overline{f}(t) \approx f(t).
\end{equation}

For the sound wave, $\rho_1$, $P_1$, and $T_1$ oscillate in-phase with one another and out-of-phase with $\vvec_1$.  Thus, for $g,h \in \left\{\rho_1, T_1, P_1 \right\}$, we have that
\begin{align}
\overline{g\, \vvec_1} &\approx 0 \\  
\overline{g\, \frac{\partial h}{\partial t}} &\approx 0.
\end{align}

Following Koulakis et al. (2018), we further assume that to zeroth order, the gas is $stationary$ and \textit{in pressure equilibrium} with the bulb and its surroundings.  This yields
\begin{align}
P_0 &= \mathrm{constant} \\ 
\vgrad P_0 &= 0 \\
\vvec_0 &= 0.
\end{align}
Finally, we use the overbar notation and brackets to indicate the time-averaged value of a second order quantity, namely
\begin{equation}
\overline{f} \equiv \langle f_2 \rangle,
\end{equation}
where the averaging is carried out over several acoustic periods (yielding a zero average for linear quantities associated with the acoustic field).

Our goal is to find an expression for $\partial\overline{T}/\partial t$.


\subsection{Momentum}
Rather than construct an equation for 2nd-order dynamics, we filter the terms of Equation \ref{eq:momentum} in time to develop a description of the slow dynamics.  This procedure retains higher-order terms, but as we show, the only higher-order term to survive the filtering operation is the advective operator.  For the time derivative term, we have
%\begin{equation}
%\label{eq:momentum}
%\rho\frac{\dv}{\dt}+\vvec\cdot\vgrad\vvec = -\vgrad P + {\color{blue} \vgrad\cdot\Dscript},
%\end{equation}

\begin{equation}
\label{eq:tder}
\langle\rho\frac{\dv}{\dt} \rangle = {\color{blue} \rhobar\langle\frac{\dv_1}{\dt}\rangle} +\rhobar\frac{\partial\overline{\vvec}_2}{\dt}  +\langle\rho_1\frac{\dv_1}{\dt}\rangle  +{\color{blue} \langle\rho_1\rangle \frac{\partial \overline{\vvec}_2}{\dt} },
\end{equation}
where the terms colored with blue are approximately zero following the temporal filtering operation.  The advective operator is similarly decomposed as
\begin{align}
\rho\vvec\cdot\vgrad\vvec &= \rhobaro\left( 
                             \langle\vvec_1\cdot\vgrad\vvec_1 \rangle 
                            +{\color{blue}\langle\vvec_1\rangle\cdot\vgrad\overline{\vvec}_2  }  
                            +{\color{blue}\overline{\vvec}_2\cdot\vgrad\langle\vvec_1\rangle  }
                            +\overline{\vvec}_2\cdot\vgrad\overline{\vvec}_2
\right)\\
&+{\color{blue} \langle\rho_1\vvec_1\cdot\vgrad\vvec_1\rangle}
                            +{\color{blue} \langle\rho_1\vvec_1\rangle\cdot\vgrad\vveco_2}
                            +{\color{blue} \vveco_2\cdot\langle\rho_1\cdot\vgrad\overline{\vvec}_1\rangle}
                            +{\color{blue} \langle\rho_1\rangle\vveco_2\cdot\vgrad\vveco_2}.
\end{align}
The pressure term easily separates as:
\begin{equation}
\vgrad P = {\color{blue} \langle \vgrad P_1 \rangle} + \vgrad\overline{P}.
\end{equation}
We assumed earlier that $P_0$ was essentially constant, but the slow dynamics, which are essentially incompressible, will induce a pressure response (i.e., a $P_2$), and so we retain $\vgrad\overline{P}$.

Our filtered momentum equation is then
\begin{equation}
\label{eq:filmom}
\rhobaro\frac{\partial\overline{\vvec}_2}{\dt} +\rhobaro\overline{\vvec}_2\cdot\vgrad\overline{\vvec}_2 = - \langle\rho_1\frac{\dv_1}{\dt}\rangle 
                             -\rhobaro\langle\vvec_1\cdot\vgrad\vvec_1 \rangle
                             -\vgrad\overline{P}
\end{equation}
Substituting Equation \ref{eq:firstgrad} in \ref{eq:filmom}, we have that
\begin{equation}
\label{eq:filtwo}
\rhobaro\frac{\partial\overline{\vvec}_2}{\dt} +\rhobaro\overline{\vvec}_2\cdot\vgrad\overline{\vvec}_2 = 
                             - \langle\rho_1\frac{\dv_1}{\dt}\rangle 
                             -\vgrad\overline{P}
 -\frac{\rhobaro}{2}\langle\vgrad v_1^2\rangle -\langle v_1^2\rangle\vgrad\rho_0 
 +\langle\vvec_1\left(\vvec_1\cdot\vgrad\rho_0\right)\rangle .
\end{equation}
Finally, combining Equations \ref{eq:linsound} and \ref{eq:firstdt} with \label{eq:filtwo}, we have that
\begin{equation}
\label{eq:filthree}
\rhobaro\frac{\partial\overline{\vvec}_2}{\dt} +\rhobaro\overline{\vvec}_2\cdot\vgrad\overline{\vvec}_2 = 
                             \frac{\vgrad\langle P_1^2 \rangle}{2\rho_0 c^2}
                             -\vgrad\overline{P}
 -\frac{\rhobaro}{2}\langle\vgrad v_1^2\rangle -\langle v_1^2\rangle\vgrad\rho_0 
 +\frac{\partial}{\dt}\langle\frac{P_1\vvec_1}{c^2} - \rho_1\vvec_1 \rangle.
\end{equation}
This is identical to Koulakis et al. (2018) with the exception of the advective term.  That term appears to be the only higher-order term that survives the filtering process.


\newcommand{\rhobar}{\overline{\rho}}
\newcommand{\Pbar}{\overline{P}}
\newcommand{\vvec}{\boldsymbol{v}}
\newcommand{\vvecbar}{\overline{\boldsymbol{v}}}
\newcommand{\rhat}{\boldsymbol{\hat{r}}}
\newcommand{\shat}{\boldsymbol{\hat{s}}}
\newcommand{\vgrad}{\boldsymbol{\nabla}}

%%%%%%%%%%%%%%%%%%%%%%%%%%%%%%%%%%%%%%%%%%%%%%%%%%%%%%5
% Base Equtations

\section{Base Equations}
The momentum equation as per Koulakis et al. (2018) is:

\begin{equation}
\label{eq:momentum}
\rhobar\frac{ \partial \vvecbar }{\partial t}  =    -\boldsymbol{\nabla}\Pbar  % pressure
						        - \frac{\rhobar}{2}\boldsymbol{\nabla}\langle v_1^2\rangle
							- \langle v_1^2\rangle\boldsymbol{\nabla}\rhobar
							+ f(r,t)
\end{equation}
where $f(r,t)$ represents the spherically symmetric, time-varying component of the pycnoclinic acoustic force, and is given by
\begin{equation}
\label{eq:dtterms}
f(r,t) \equiv \frac{\boldsymbol{\nabla}\langle P_1^2 \rangle}{2\rhobar c^2}+\frac{\partial}{\partial t}\left\langle \frac{P_1\boldsymbol{v_1}}{c^2} -\rho_1 \boldsymbol{v_1} \right\rangle.
\end{equation}
Here, $\rho$ is the plasma density, $P$ is the plasma pressure, $\vvec$ is the plasma velocity, and $c$ is the plasma sound speed.  Quantities with a subscript $1$ represent those associated with the sound wave when the system is exposed to microwave radiation pulsed in-resonance with its fundamental model.  Angular brackets denote averaging over several sound acoustic periods.  The velocity induced by the pycnoclinic effect is denoted by $\vvecbar$.  The quantities $\rhobar$ and $\Pbar$ indicate the time-averaged, spatially varying background density and pressure respectively.   They take the form
\begin{equation}
\rhobar = \rho_g+\rho_c(r)
\end{equation}
and
\begin{equation}
\Pbar = P_g+P_c(r)
\end{equation}
The density and Pressure in the absence of a resonant sound wave are denoted by $\rho_g$ and $P_g$ respectively; they are taken to be constant.   Compressibility effects due to the pycnoclinic effect are denoted by the subscripts $c$.   Koulakis et al. (2018) observe a 1\% compression of the plasma at the center, implying that
\begin{equation}
\label{eq:compress}
\rho_g \gg \rho_c \,\,\,\mathrm{and}\,\,\,
\frac{\rho_c}{\rho_g}\sim \mathrm{O}(10^{-2}).
\end{equation}

%%%%%%%%%%%%%%%%%%%%%%%%%%%%%%%%%%%%%%%%%%%%%%
% Modified Equations

\section{Modifications to Base Equations}
Before we can numerically model the dynamical effects of the pycnoclinic acoustic force, we make some modifications to these equations, highlighted in blue and discussed below.  Our modified momentum equation is
\begin{equation}
    \rho\frac{ \partial \vvecbar }{\partial t}   =  {\color{blue} - \rho\vvecbar\cdot\boldsymbol{\nabla}\vvecbar}
    {\color{blue} \,-\, 2\rho\Omega\boldsymbol{\hat{z}}\times\vvec  + \rho\Omega^2r\boldsymbol{\hat{s}}}
    {\color{blue} + \boldsymbol{\nabla}\cdot\boldsymbol{\mathcal{D}}}
    -\boldsymbol{\nabla}P  % pressure
						            - \frac{\rho}{2}\boldsymbol{\nabla}\langle v_1^2\rangle
							    - \langle v_1^2\rangle\boldsymbol{\nabla}\rho
							    + f(r,t),
\end{equation}
where $\boldsymbol{\hat{s}}$ is a unit vector in the direction of cylindrical radius, and $\Omega$ is the rotation rate of the bulb. We have ommited overbars on the density and pressure, which now include a small fluctuating component allowed to vary spatially and with time, namely
\begin{equation}
\rho = \rho_g+\rho_c(r)+\tilde{\rho}(\boldsymbol{r},t)
\end{equation}
and
\begin{equation}
P = P_g+P_c(r)+\tilde{P}(\boldsymbol{r},t),
\end{equation}
subject to the constraints that
\begin{equation}
\rhobar \,\gg\, \tilde{\rho},
\end{equation}
and
\begin{equation}
\Pbar \,\gg\, \tilde{P}.
\end{equation}

The viscous stress tensor $\boldsymbol{\mathcal{D}}$ is given by
\begin{equation}
\mathcal{D}_{ij} = 2\rho\nu\left[e_{ij}-\frac{1}{3}\boldsymbol{\nabla}\cdot\boldsymbol{v}\right],
\end{equation}
where $\nu$ is the kinematic viscosity (assumed to be a constant) and $e_{ij}$ is the strain-rate tensor.  

Briefly, we discuss the reason for including these new terms.
\begin{enumerate}
\item {\bf Fluctuating density and pressure.}   Equations \ref{eq:momentum} and \ref{eq:dtterms} define a system subject to forces that are radial and spherically symmetric in nature.  The plumes that erupt periodically during the experiment, however, exhibit significant deviations from spherical symmetry.  If we want to explore those dynamics, we cannot neglect small deviations in density and pressure from a spherically-symmetric, steady background state.  This approach is similar to that adopted when studying convection under Bousinesq approximation; such studies include a formally negligible fluctuating buoyancy term.  
\item {\bf Advection. }  This term should be small, but we retain it in case the ejected plumes convey significant momentum in addition to heat. 
\item {\bf Viscosity. }  This addition is a practical necessity if we are to model this system numerically.  Moreover, the remark by Jon that the system Ekman number is O(10$^{-5}$) suggests that viscous effects may be non-negligible in this system.
\item {\bf Rotation. }  Koulakis et al. (2018) suggest that the amplitude of the centrifugal force resulting from the bulb's spin is about 10\% that of the pycnoclinic acoustic force.   This suggests that rotation may play some role in the observed dynamics.
\end{enumerate}


%%%%%%%%%%%%%%%%%%%%%%%%%%%%%%%%%%%%%%%%%%%%%%%%%%%5
% Model Equations

\section{The Model Equations}
Our goal is to now construct an appropriate set nomdimensional equation set to model with Rayleigh.  We begin with the continuity equation
\begin{equation}
\frac{\partial \rho}{\partial t} = -\boldsymbol{\nabla}\cdot\left(\rho\vvecbar\right),
\end{equation}
which can be rewritten as
\begin{equation}
\boldsymbol{\nabla}\cdot\vvecbar =-\frac{1}{\rho}\frac{\mathrm{D}\tilde{\rho}}{\mathrm{D}t}-\frac{\overline{v_r}}{\rho}\frac{\partial \rho_c}{\partial r} \approx -\frac{1}{\rho_g}\frac{\mathrm{D}\tilde{\rho}}{\mathrm{D}t}-\frac{\overline{v_r}}{\rho_g}\frac{\partial \rho_c}{\partial r},
\end{equation}
where the advective derivative is defined as
\begin{equation}
\frac{D}{Dt}\equiv\frac{\partial}{\partial t} + \vvecbar\cdot\vgrad.
\end{equation}
As $\tilde{\rho}$ and $\rho_c$ are both assumed small relative to $\rho_g$, the system is approximately incompressible, leaving us with
\begin{equation}
\label{eq:continuity}
\vgrad\cdot\vvecbar = 0.
\end{equation}
Note that this implies that {\em there can be spherically symmetric component of radial velocity $\overline{v_r}$}.  By making the approximation \ref{eq:continuity}, we thus confine ourselves to considering a system with a steady background state has reached a steady-state pseudo-equilibrium.   We could work in the anelastic approximation if we wanted to including radial variation of the background state, but since the compression factor is around 1\%,  compressibility effects should be negligible.

Dividing equation \ref{eq:momentum} by $\rho$, and applying similar reasoning, we arrive at the momentum equation
\begin{equation}
\label{eq:momnew}
\frac{ D \vvecbar }{D t}   \approx  
\,-\, 2\Omega\boldsymbol{\hat{z}}\times\vvecbar 
-\frac{1}{\rho_g}\boldsymbol{\nabla}\tilde{P}  % pressure
-\left(\frac{\tilde{\rho}}{\rho_g}+1\right)\Omega^2r\boldsymbol{\hat{s}}
						        - \frac{1}{2}\left( 1 + \frac{\tilde{\rho}}{\rho_g}\right)\boldsymbol{\nabla}\langle v_1^2\rangle
							- \langle v_1^2\rangle\boldsymbol{\nabla}\left(\frac{\tilde{\rho}}{\rho_g}\right)
+ \nu\boldsymbol{\nabla}^2\vvecbar
+\frac{1}{\rho_g}f(r,t)
\end{equation}

We now subtract the mean state, which describes hydrostatic balance in the radial direction and pressure-centrifugal balance in the theta direction.   For hydrostatic balance in the radial direction, we arrive at
\begin{equation}
\label{eq:hydrostatic}
0   =  - \left[\vvecbar\cdot\boldsymbol{\nabla}\vvecbar\right]_{00}
- \frac{2}{\pi}\Omega^2r
-\frac{1}{\rho_g}\frac{\partial \tilde{P}_{00}}{\partial r}  % pressure
						        - \frac{\tilde{\rho}_{00}}{2\rho_g}\frac{\partial }{\partial r}\langle v_1^2\rangle
                                - \frac{1}{2}\boldsymbol{\nabla}\langle v_1^2\rangle
							- \langle v_1^2\rangle\boldsymbol{\nabla}\tilde{\rho}_{00}
							+ f(r,t),
\end{equation}
where the spherically-symmetric ($\ell=0$, $m=0$) component of denoted by the subscript 00.  This is a diagnostic equation for the mean pressure perturbation, and has no impact on the dynamics due to the lack of a spherically symmetric component of $\overline{v}_r$.  In the ${\hat{\theta}}$-direction, a portion of the pressure gradient must be sufficiently strong enough to balance the large, non-buoyancy-related portion of the centifugal term.  Denoting that portion by an overdot, we have
\begin{equation}
\label{eq:twind}
0 = \frac{\partial }{\partial\theta}\dot{{P}} -\mathrm{cos}\,\theta\Omega^2r.
\end{equation}

Subtracting Equations \ref{eq:hydrostatic} and \ref{eq:twind} from the Equation \ref{eq:momnew}, we arrive at the momentum equation describing the dynamics, namely
\begin{equation}
\frac{ D \vvecbar }{D t}   =  
\,-\, 2\Omega\boldsymbol{\hat{z}}\times\vvecbar 
-\frac{1}{\rho_g}\boldsymbol{\nabla}P'  % pressure
-\frac{\rho'}{\rho_g}\Omega^2r\boldsymbol{\hat{s}}
						        - \frac{1}{2}\frac{\rho'}{\rho_g}\boldsymbol{\nabla}\langle v_1^2\rangle
							- \langle v_1^2\rangle\boldsymbol{\nabla}\left(\frac{\rho'}{\rho_g}\right)
+ \nu\boldsymbol{\nabla}^2\vvecbar
\end{equation}
Here, primed quantities indicated fluctuations about the hydrostatic and centrifugal-balance background density and pressure, such that
\begin{equation}
\rho' = \tilde{\rho}-\tilde{\rho}_{00},
\end{equation}
and
\begin{equation}
P' = \tilde{P}-\tilde{P}_{00}-\dot{P}.
\end{equation}

We can further define an effective gravity vector $\boldsymbol{g}$, produced by the pycnoclinic effect centrifugal force, as
\begin{equation}
\boldsymbol{g} = \frac{1}{2}\frac{\partial}{\partial r}\langle v_1^2\rangle\rhat + \Omega^2 r\boldsymbol\shat  .
\end{equation}

Finally, it is standard practice to write the relative density perturbation in terms of the associated temperature perturbation and the coefficient of thermal expansion $\alpha$ as
\begin{equation}
\label{eq:texp}
\frac{\rho'}{\rho_g}=-\alpha T'.
\end{equation}

The final form of our momentum equation then resembles the standard form, with the extra addition of the pycnoclinic term, namely
\begin{equation}
\frac{ D \vvecbar }{D t}   =  
\,-\, 2\Omega\boldsymbol{\hat{z}}\times\vvecbar 
-\frac{1}{\rho_g}\boldsymbol{\nabla}P'  % pressure
						        +\alpha T'\boldsymbol{g}
							+\alpha \langle v_1^2\rangle\boldsymbol{\nabla}T'
+ \nu\boldsymbol{\nabla}^2\vvecbar
\end{equation}

\section{Nondimensional Equations and Parameters}
We nondimensionalize this system by choosing the shell depth $L$ and viscous diffusion time $L^2/\nu$ as the characteristic length and time scales.  We adopt nondimensional temperature scale of $\Theta$ (to be determined) and a characteristic gravitational acceleration $g_0$, to be
\begin{equation}
g_0 = \frac{v_0^2}{2 L},
\end{equation}
where 
\begin{equation}
v_0^2 \equiv \mathrm{max}\left\langle v_1^2 \right\rangle.
\end{equation}
Finally, we choose the customary scaling of $\Omega\nu\rho_g$ for our nondimensional pressure scale.  Multiplying by $\frac{L^3}{\nu^2}$ then yields the nondimensional system
\begin{equation}
\label{eq:nondim}
\frac{ D \vvec }{Dt}   = 
\,-\, \frac{2}{E}{\hat{z}}\times\vvec  
-\frac{1}{E}\boldsymbol{\nabla}P'  % pressure
						        + \frac{\mathrm{Ra}}{\mathrm{Pr}} T' \boldsymbol{\tilde{g}}
							+2\frac{\mathrm{Ra}}{\mathrm{Pr}}\langle v_1^2\rangle\boldsymbol{\nabla} T'
+ \boldsymbol{\nabla}^2\vvec.
\end{equation}
Here, $\boldsymbol{\tilde{g}}$ denotes nondimensional gravitational accelation, given by
\begin{equation}
\boldsymbol{\tilde{g}} = \frac{\partial}{\partial r}\langle v_1^2\rangle\rhat + \mathrm{Fr}\, r\boldsymbol\shat,
\end{equation}
where Fr is a Froud number, expressing the relative important of centrifugal buoyancy to pycnoclinic buoyancy.  It is given by
\begin{equation}
\mathrm{Fr} = \frac{2\Omega^2 L^2}{v_0^2}.
\end{equation}

Three additional nondimensional numbers appear in Equation \ref{eq:nondim}.  The first is the Rayleigh number, $Ra$, given by
\begin{equation}
\mathrm{Ra} = \frac{\alpha g_0 \Theta L^3}{\nu\kappa},
\end{equation}
where $\kappa$ is the thermal diffusivity (assumed to be constant as with $\nu$).  The Rayleigh number represents 
relative importance of buoyancy to diffusion in the system.  It can be written as the product
\begin{equation}
Ra=\left(\frac{\tau_{\nu}}{\tau_{\mathrm{ff}}}\right)\left(\frac{\tau_{\kappa}}{\tau_{\mathrm{ff}}}\right),
\end{equation} 
where $\tau_{\nu}$, $\tau_{\kappa}$, and $\tau_{\mathrm{ff}}$ respetively denote the viscous, thermal, and free-fall timescales associated with the fluid layer.

The second nondimensional number that appears is the Ekman number, E, given by
\begin{equation}
\mathrm{E} = \frac{\nu}{\Omega L^2} = \frac{\tau_\Omega}{\tau_\nu}.
\end{equation}
It expresses the ratio of the rotation period $\tau_\Omega$ to that viscous timescale $\tau_\nu$.

Finally, the Prandtl number, Pr represents the relative strengths of viscous and thermal diffusion.  It is given by
\begin{equation}
\mathrm{Pr} = \frac{\nu}{\kappa}.
\end{equation}

In order to evolve this system, we need a nondimensional continuity equation;  its form remains unchanged from that in \ref{eq:continuity} (i.e.,  no nondimensional parameters appear in the continuity equtation).  Finally, we need an equation for the internal energy of the system.  Given that energy is deposited internally via microwave absorption, we adopt the standard form of the nondimensional internal energy equation
\begin{equation}
\label{eq:te}
\frac{D\tilde{T}}{Dt}=-\frac{1}{Pr}\nabla^2\tilde{T} + Q,
\end{equation}
where Q is a constant function of space and time and represents the nondimensional thermal energy input.  Note that we have not separated out the $\ell=0$,$m=0$ component of temperature here;  the mean state evolves dynamically through both conduction and advection.

Equations \ref{eq:continuity}, \ref{eq:nondim}, and \ref{eq:te}  fully describe the system we wish to evolve numerically.  They are what has now been implemented into the Rayleigh code.

\subsection{Nondimensional Values}
Now that we have formulated our nondimensional equations, it remains for us to establish appropriate values for our nondimensional parameters.
In this experiment, $L \approx 0.015$ m, $\Omega \approx 314 \,\,\mathrm{ s}^{-1}$, and $v_0 \approx 20 \,\,\mathrm{m}\,\,\mathrm{s}^{-1}$.  

This immeidately yields Fr=0.111.  Centrifugal buoyancy is non-negligible at this rotation rate.

Establishing the other values is more difficult without knowing the value of our diffusion coefficients.  Jon, however, relayed that from an earlier conversation with Seth, he believes that $E\approx 10^{-5}$.   This yields a ballpark value of the viscosity of $\nu\approx 7\times 10^{-7}$ m$^2$ s$^{-1}$.  From Equations \ref{eq:compress} and \ref{eq:texp} we can estimate an upper bound on the characteristic temperature $\Theta$, arriving at $\alpha\Theta\leq 0.01$.   

This yields $\frac{Ra}{Pr}\leq 9\times 10^{8}$.   

Depending on the value of $Pr$, this would appear prohibitively high (computationally), but I close by noting two points.  (1)  The appropriate value for $\Theta$ should be something that is characteristic of perturbations about the spherically symmetric background state, and those perturbations should be much less than $\rho_c$'s value of 0.01.  (2)  Having seen the system in person, it is clearly not high-Rayleigh-number convection, which lends credence to point 1.   I suspect diminishing the estimate above by an additional factor of 10-100 is reasonable, placing this system well into the computationally tractable regime.  Though, this does depend somewhat on the value of Pr.
 

\clearpage

%%%%%%%%%%%%%%%%%%%% END


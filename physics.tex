\clearpage
\newcommand{\rhobar}{\overline{\rho}}
\newcommand{\vvec}{\boldsymbol{v}}
\newcommand{\rhat}{\boldsymbol{\hat{r}}}
\newcommand{\shat}{\boldsymbol{\hat{s}}}
\section{Physics Controls}\label{sec:physics}
Words

\subsection{PAF Equations)}

The momentum equation as per Koulakis et al. (2018).

\begin{align*}
\rhobar\frac{ \partial \boldsymbol{v} }{\partial t}  &=    -\boldsymbol{\nabla}P  % pressure
						        - \frac{\rhobar}{2}\boldsymbol{\nabla}\langle v_1^2\rangle
							- \langle v_1^2\rangle\boldsymbol{\nabla}\rhobar
							+ f(r,t) \\
f(r,t) &\equiv \frac{\boldsymbol{\nabla}\langle P_1^2 \rangle}{2\rhobar c^2}+\frac{\partial}{\partial t}\left\langle \frac{P_1\boldsymbol{v_1}}{c^2} -\rhobar_1 \boldsymbol{v_1} \right\rangle \\
\rhobar &= \rho_g+\rho_c(r)
\end{align*}
Where $\rho_g$ is the plasma density in the absence of compression due to the pycnoclinic effect, assumed to be constant.   Compression due to the pycnoclinic effect is represented by $\rho_c$.   Koulakis et al. (2018) observe a 1\% compression of the plasma at the center, implying that:
\begin{align*}
\rho_g \gg \rho_c \,\,\,\mathrm{and}\,\,\,
\frac{\rho_g}{\rho_0}\sim \mathrm{O}(10^{-2})
\end{align*}

We now make some modifications to these equations, which we highlight in blue and discuss below.  The modified momentum equation is
\begin{equation}
\rho\frac{ \partial \vvec }{\partial t}   =  {\color{blue} - \rho\vvec\cdot\boldsymbol{\nabla}\vvec}
{\color{blue} \,-\, 2\rho\Omega\boldsymbol{\hat{z}}\times\vvec  - \rho\Omega^2r\boldsymbol{\hat{s}}}
{\color{blue} + \boldsymbol{\nabla}\cdot\boldsymbol{\mathcal{D}}}
-\boldsymbol{\nabla}P  % pressure
						        - \frac{\rho}{2}\boldsymbol{\nabla}\langle v_1^2\rangle
							- \langle v_1^2\rangle\boldsymbol{\nabla}\rho
							+ f(r,t),
\end{equation}
where $\boldsymbol{\hat{s}}$ is a unit vector in the direction of cylindrical radius, and where the density now includes a small fluctuating component such that
\begin{equation}
\rho = \rhobar +{\color{blue} \rho'(\boldsymbol{r},t)}
\end{equation}
and
\begin{equation}
\rhobar \,\gg\, \rho'.
\end{equation}
The viscous stress tensor $\boldsymbol{\mathcal{D}}$ is given by
\begin{equation}
\mathcal{D}_{ij} = 2\rho\nu\left[e_{ij}-\frac{1}{3}\boldsymbol{\nabla}\cdot\boldsymbol{v}\right],
\end{equation}
where $\nu$ is the kinematic viscosity (assumed to be a constant) and $e_{ij}$ is the strain-rate tensor.  

Briefly, we discuss the reason for including these new terms.
\begin{enumerate}
\item {\bf Viscosity. }  This addition is a practical necessity if we are to model this system numerically.  Moreover, the suggestion by Jon that the system Ekman number is O(10$^{-5}$) suggests that viscous effects may be non-negligible in this system.
\item {\bf Rotation. }  Koulakis et al. (2018) suggest that the amplitude of the centrifugal force resulting from the bulb's spin is about 10\% that of the pycnoclinic force.   This suggests that rotation may play a role in the observed dynamics.
\item {\bf Fluctuating density.}  Much as in the Bousinesq approximation, this term is formally negligible, but necessary if we want to study buoyancy-driven dynamics.  
\item {\bf Advection. }  This term should be small, but, as with the fluctuating density, we retain it for the dynamics. 
\end{enumerate}

Before moving on, let us examine the continuity equation.   We begin with
\begin{equation}
\frac{\partial \rho}{\partial t} = -\boldsymbol{\nabla}\cdot\left(\rho\vvec\right),
\end{equation}
which simplifies to 
\begin{equation}
\frac{1}{\rho_0}\frac{\mathrm{D}\rho'}{\mathrm{D}t}-\frac{v_r}{\rho_0}\frac{\partial \rho_c}{\partial r} = -\boldsymbol{\nabla}\cdot\vvec.
\end{equation}
As $\rho'$ and $\rho_c$ are both assumed small relative to $\rho_g$, the system is approximately incompressible, namely
\begin{equation}
\boldsymbol{\nabla}\cdot\vvec = 0.
\end{equation}
Note that this means {\em there is no spherically symmetric component of the radial velocity}.

Next, let us subtract the spherically symmetric state, denoting spherical means by a subscript 0.   While the fluctuating density can in principle possess a spherically symmetric component, it cannot contribute to the dynamics. We thus ommit this term, and use the ' to denote deviations about the spherically symmetric state moving forward.

Substracting out the mean state leads to the following system of equations:
\begin{equation}
\label{hydrostatic}
0   =  - \left[\rho\vvec\cdot\boldsymbol{\nabla}\vvec\right]_0
- [\rho_0\Omega^2r\boldsymbol{\hat{s}}]_0
-\boldsymbol{\nabla}P_0  % pressure
						        - \frac{\rho_0}{2}\boldsymbol{\nabla}\langle v_1^2\rangle
							- \langle v_1^2\rangle\boldsymbol{\nabla}\rho_0
							+ f(r,t),
\end{equation}
\begin{equation}
\label{momentum}
\rho_0\frac{ \partial \vvec }{\partial t}   =  - \rho_0\vvec\cdot\boldsymbol{\nabla}\vvec
\,-\, 2\Omega\rho_0\boldsymbol{\hat{z}}\times\vvec  
-\boldsymbol{\nabla}P'  % pressure
-\rho'\Omega^2r\boldsymbol{\hat{s}}
						        - \frac{\rho'}{2}\boldsymbol{\nabla}\langle v_1^2\rangle
							- \langle v_1^2\rangle\boldsymbol{\nabla}\rho'
+ \nu\boldsymbol{\nabla}^2\vvec
\end{equation}

Equation \ref{hydrostatic} provides a diagnostic equation for $P_0$ and represents the effective hydrostatic balance achieved in the system.  Equation \ref{momentum} describes the dynamics of the fluctuations, which we simply further by dividing by $\rho_0$ and rewriting the fluctuating density in terms of the temperature.  This yields
\begin{equation}
\frac{ \partial \vvec }{\partial t}   =  - \vvec\cdot\boldsymbol{\nabla}\vvec
\,-\, 2\Omega\boldsymbol{\hat{z}}\times\vvec  
-\boldsymbol{\nabla}P'  % pressure
						        + \alpha T' \boldsymbol{g}
							+\alpha \langle v_1^2\rangle\boldsymbol{\nabla} T'
+ \nu\boldsymbol{\nabla}^2\vvec,
\end{equation}

where $\boldsymbol{g}$ is the effecti gravity produced by the pycnoclinic effect centrifugal force, namely
\begin{equation}
\boldsymbol{g} = \frac{1}{2}\frac{\partial}{\partial r}\langle v_1^2\rangle\rhat + \Omega^2 r\boldsymbol\shat  .
\end{equation}
We nondimensionalize this system by choosing the shell depth $L$ and viscous diffusion time $L^2/\nu$ as the characteristic length and time scales.   We further take the characteristic graviational acceleration to be $g_0$, given by
\begin{equation}
g_0 = \frac{v_0^2}{2 L},
\end{equation}
where 
\begin{equation}
v_0^2 \equiv \mathrm{max}\left\langle v_1^2 \right\rangle.
\end{equation}
We choose $\tilde{P}$ and $\tilde{T}$ as our pressure and temperature scales respectively and
nondimensionalize by multiplying by $\frac{L^3}{\nu}$.  This yields
\begin{equation}
\frac{ \partial \vvec }{\partial t}   =  - \vvec\cdot\boldsymbol{\nabla}\vvec
\,-\, \frac{2}{E}{\hat{z}}\times\vvec  
-\frac{L^3}{\nu^2}\boldsymbol{\nabla}P'  % pressure
						        + \frac{\mathrm{Ra}}{\mathrm{Pr}} T' \boldsymbol{\tilde{g}}
							+2\frac{\mathrm{Ra}}{\mathrm{Pr}}\langle v_1^2\rangle\boldsymbol{\nabla} T'
+ \boldsymbol{\nabla}^2\vvec.
\end{equation}
{\bf P is still al ittle odd -- I get 1/E2}.
The Rayleigh number Ra is given by
\begin{equation}
\mathrm{Ra} = \frac{\alpha g_0 \tilde{T} L^3}{\nu\kappa},
\end{equation}
the Ekman number E by
\begin{equation}
\mathrm{E} = \frac{\nu}{\Omega L^2},
\end{equation}
and Prandtl number Pr by
\begin{equation}
\mathrm{Pr} = \frac{\nu}{\kappa}.
\end{equation}
Finally, the nondimensional gravitational accelation vector is given by
\begin{equation}
\boldsymbol{\tilde{g}} = \frac{\partial}{\partial r}\langle v_1^2\rangle\rhat + \mathrm{Fr}\, r\boldsymbol\shat,
\end{equation}
where Fr a Froud number.  In this case, it expresses the relative important of centrifugal buoyancy to pycnoclinic buoyancy, and is given by
\begin{equation}
\mathrm{Fr} = \frac{2\Omega^2 L^2}{v_0^2}.
\end{equation} 
In this experiment, $L \approx 0.03$ m, $\Omega \approx 314 \,\,\mathrm{ s}^{-1}$, and $v_0 \approx 20 \,\,\mathrm{m}\,\,\mathrm{s}^{-1}$.  This yields Fr=0.443.  Centrifugal buoyancy is thus likely important for the dynamics at this rotation rate.
\clearpage

%%%%%%%%%%%%%%%%%%%% END


\section{First Order Equations}
We first describe the rapidly-oscillating, first-order motions associated with an adiabatic sound wave. The notes here are taken from ( and redundant with) K18, but they are useful to include in this same document.  Following K18, we first separate the first order terms from Equations \ref{eq:continuity} and \ref{eq:momentum}, giving us
\begin{equation}
\label{eq:firstmom}
\rho_0\frac{\dv_1}{\dt} = -\vgrad P_1,
\end{equation}
and
\begin{equation}
\label{eq:firstcont}
\frac{\drho_1}{\dt} + \rho_0\vgrad\cdot\vvec_1 + \vvec_1\vgrad\rho_0 = 0.
\end{equation}

The acoustic motions are assumed to be adiabatic and, given our earlier discussion from THERMOSECTION, they must satisify the relationship
\begin{equation}
\label{eq:sound}
\frac{\Drho}{\Dt} = \frac{1}{c^2}\frac{\DP}{\Dt},
\end{equation}
where $c$ is the sound speed.   Recalling that $P_0$ is assumed to be constant, at first order, Equation \ref{eq:linsound} gives
\begin{equation}
\label{eq:linsound}
\frac{\drho_1}{\dt} +\vvec_1\cdot\vgrad\rho_0 = \frac{1}{c^2}\frac{\dP_1}{\dt}.
\end{equation}

\subsection{Useful Relationships}
We note a few useful relationships here.  Equation \ref{eq:firstmom} implies that 
\begin{equation}
\label{eq:firstcurl}
\vgrad\times\left(\rho_0\vvec_1\right) = 0,
\end{equation}
which as shown in K18, allows us to write
\begin{equation}
\label{eq:firstgrad}
-\vvec_1\cdot\vgrad\vvec_1 = -\frac{1}{2}\vgrad v_1^2 -\frac{1}{\rho_0}v_1^2\vgrad\rho_0 +\frac{1}{\rho_0}\vvec_1\left(\vvec_1\cdot\vgrad\rho_0\right).
\end{equation}
Equation \ref{eq:firstcont} implies that
\begin{equation}
\label{eq:firstdt}
P_1\frac{\dv_1}{\dt} = -\frac{1}{2\rho_0}\vgrad P_1^2.
\end{equation}



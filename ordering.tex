\section{Problem Setup}
We follow Koulakis et al (2018) and seek a solution that is comprised of a rapidly-oscillating, high-amplitude sound wave, and a low-amplitude, slowly varying flow.   The approach here differs slightly from theirs.  Rather than developing a set of equations that are second-order in the perturbations, we choose to filter our the rapid acoustic motions to obtain a higher-order set of equations (possibly greater than second-order).

\subsection{Ordering Assumptions}

We write the thermodynamic variables $\rho$, $P$, and $T$ as
\begin{equation}
f(r,\theta,\phi,t) = f_0(r,\theta,\phi,t) + f_1(r,\theta,\phi,t).
\end{equation}
A subscript $0$ denotes the ambient gas in the absence of sound.  We assume that $P_0$ is a constant, but that $\rho_0$ and $T_0$ can vary in space and (slowly) in time.  A subscript $1$ denotes variations associated with the high-amplitude sound field, assumed to be small, such that that
\begin{equation}
f_0 \gg f_1 .
\end{equation}
First-order components of the thermodynamic variables are assumed to vary periodicaly and much more rapidly than the zeroth-order components, so that
\begin{equation}
\frac{\partial f_1}{\dt} \gg \frac{\partial f_0}{\dt}.
\end{equation}




